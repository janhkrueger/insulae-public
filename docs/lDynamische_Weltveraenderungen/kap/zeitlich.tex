\chapter{Zeitlich begrenzte Ereignisse}
Unter die im Titel genannten Ereignisse wird all jenes eingereiht, was \newline
TODO: Beschreibung noch nicht fertig. Viel zu sp{\"a}t nun.

\section {Auftreten besonderer Rohstoffe}
F{\"u}r einen begrenzten Zeitraum entstehen auf zuf{\"a}llig ausgew{\"a}hlten
leeren Feldern seltene Hotspots sonst nicht vorkommender Rohstoffe. Prinzipiell
kann mittels Aufruf aus dem Verwaltungsportal jeder Rohstoff so platziert
werden. Es ist jedoch zu beachten das nicht f{\"u}r jeden Rohstoff entsprechende
Bilder vorliegen. Von der Glaubw{\"u}rdigkeit mancher Kombinationen einmal ganz
abgesehen.
\newline\newline \textbf{Skriptname:} gimmegoods.py \newline\newline
\textbf{Aufrufparameter:} InselID, RohstoffID, Dauer
\newline\newline
Wird kein Parameter wird angegeben, verarbeitet das Skript jede Insel und
w{\"a}hlt zuf{\"a}llig Rohstoff und Dauer aus. Hierbei wird die hinterlegte
Positivliste angewendet.

\section {Wandernde H{\"a}ndler}
Innerhalb der Welt von Insulae gibt es H{\"a}ndler welche nicht fest an einen
Ort festgelegt sind, sondern im Lauf der Zeit ueber die Inseln wandern. Diese
Wanderungen werden in regelm{\"a}\ss igen Abst{\"a}nden ausgef{\"u}hrt. Eine
manuelle Wanderung ist hierbei m{\"o}glich.\newline\newline
Wandernde H{\"a}ndler bieten die Chance bestimmte Rohstoffe / Rezepte,
Gegenst{\"a}nde mit einem Seltenheitsfaktor zu versehen. Diese
H{\"a}ndler k{\"o}nnen in jedem Gebiet auftauchen, auch wenn
dieses f{\"u}r manche Spieler nicht erreichbar ist. Es liegt hier
bei den Spielern, wie nach Auftauchen eines H{\"a}ndlers
verfahren wird, ob der aktuelle Ort geheim bleibt oder die
Mitspieler darauf hingewiesen werden.\newline
Sollte der H{\"a}ndler bek{\"a}mpft und besiegt werden, so startet dieser nach
Ablauf einer Woche an einem zuf{\"a}lligen Puntk seiner Route. Die Koordinate
von welcher er vertrieben wurde, wird jedoch f{\"u}r ein halbes Jahr
deaktiviert. Wer den H{\"a}ndler vor die T{\"u}r setzt, muss eben nicht damit
rechnen das dieser so schnell wieder kommt.
\newline
Im Gegenzug kann es auch vorkommen, das Diebesbanden von Insel zu Insel ziehen
und dabei entsprechenden wirtschaftlichen Schaden anrichten indem Rohstoffe aus
Geb{\"a}uden verschwinden, Pferde aus Routen gestohlen werden etc.
Diese Art von wandernden H{\"a}ndlern kann jedoch nach
Entdeckung bek{\"a}mpt werden. Der betreffende H{\"a}ndler
starten dann seine Reise an einem zuf{\"a}llig ausgew{\"a}lten
Ort seiner Route nach einer Woche neu. Auf der betreffenden
Insel wird allerdings fuer ein halbes Jahr ein weiterer Punkt
erzeugt an dem diese Art von H{\"a}ndler erscheinen kann.\newline\newline
\textbf{Skriptname:}
gypsiestrampsandthieves.py
\newline\newline
\textbf{Aufrufparameter:} HaendlerID, Rotationen
\newline\newline Wird kein Parameter wird angegeben, verarbeitet das Skript jede Insel und w{\"a}hlt zuf{\"a}llig Rohstoff und Dauer aus. Hierbei wird die hinterlegte Positivliste angewendet.

\section {Ver{\"a}ndertes Monsterwachstum}
Das Monsterwachstum auf einer Insel wird stimuliert oder ausgebremst. Eine
Stimulation kann daf{\"u}r sorgen das die Spieler sich verst{\"a}rkt um die
Monster k{\"u}mmern m{\"u}ssen. Muss nur entsprechend weh tun. \newline Sofern
nichts anderes vorgegeben wird, werden die inden t{\"a}glichen Datenerhebungen
gesammelten Daten bez{\"u}glich Anzahl und St{\"a}rke der Spieler zur Bestimmung
von St{\"a}rke und Dauer des ver{\"a}nderten Wachstums herangezogen.
\newline\newline
\textbf{Skriptname:}
wewillnotgrowold.py 
\newline\newline
\textbf{Aufrufparameter:} InselID, MonsterID, Variante, Dauer
\newline\newline
Die Variante gibt an, ob es zu einem verst{\"a}rktem oder geringerem Wachstum
kommt. Wird die Variante nicht gesetzt, wird von einem Wachstum ausgegangen.
