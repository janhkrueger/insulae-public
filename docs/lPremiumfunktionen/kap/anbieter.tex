\chapter{Anbieterbetrachtung}
Unter der aktuellen Betrachtung scheint es nur sinnvoll zu sein
hier mehrere externe Bezahlsysteme einzubinden. Zum einen muss bei einem
externen System an sich keine Zahlungsverwaltung aufgebaut werden, die
Abrechnungen liegen komplett bei Dritten. Insbesondere ein Mahnverfahren ist
hierbei nicht zu beachten.\\
Andererseits sind bei einem externen System von Natur aus h{\"o}here Zahlungen
notwendig. Dies liegt an den Geb{\"u}hren welche diese Anieter verlangen. So ist
zum Beispiel eine Zahlung mittels PayPal unter 1.99 EUR nicht zu empfehlen, da
hier die Anbietergeb{\"u}hren fast komplett alles verschlingen.\\
Wochenendschn{\"a}ppchen wie Artikel fuer 99 Cent sind somit nicht realistisch,
abgesehen von Promo-Aktionen um damit auf externen Portalen zu werben.

\section {PayPal}
HINWEIS: Paypal �nderte im Jahre 2010 seine Kostenstruktur. Dies ist hier in
diesem Dokument noch nicht eingepflegt!

PayPal, ein Ableger von Ebay, wird als
der gr{\"o}sste Anbieter von Bezahlsystemen angesehen. Hier sei ein kurzer {\"U}berblick gegeben. Nach aktueller Datenlage
w{\"u}rde ich PayPal in die engere Auswahl nehmen.

Im Vergleich zu anderen Anbietern verlangt PayPal keine monatlichen Geb{\"u}hren
oder eine Einrichtungsgeb{\"u}hr. Dies macht PayPal gerade f{\"u}r Kunden die nicht
mit einem Mindestumsatz rechnen k{\"o}nnen interessant. F{\"u}r den Fall das in einem
Monat keine Kundenzahlungen anstehen, entstehen f{\"u}r den Betreiber keine
Aufw{\"a}nde.
 
Fuer jede Transaktion aus EU-L{\"a}ndern sowie Norwegen, Island und Lichtenstein
werden 0,35 EUR plus 1,9\% der Transaktionsh{\"o}he berechnet. Dies l{\"a}sst PayPal
f{\"u}r einen Kunden g{\"u}nstiger darstehen wie ClickandBuy, welche mit 2,9\%
rechnen. F{\"u}r Zahlungen ausserhalb des oben genannten Raumes werden 3,9\% an
Geb{\"u}hren verlangt. PayPal sieht allerdings auch eine Staffelung vor, bei der
die Geb{\"u}hren umso geringer werden, je h{\"o}her der Umsatz ist. Die
Geb{\"u}hrentabelle ist der PayPal-Webseite entnommen:
\begin{center}
\begin{tabular}{llllll}
{Monatlicher Umsatz} & {EU+} & {Non-EU} \\
\hline
				&		 &		   \\
bis 5.000			& 1,9\%	 & 3,9\%	   \\
5.001 - 25.000			& 1,7\%	 & 3,7\%	   \\
25.001 - 50.000			& 1,5\%	 & 3,5\%	   \\
> 50.000			& 1,2\%	 & 3,2\%	   \\
\hline
\end{tabular}
\end{center}
Es steht in jedem Fall die Pauschale von 0,35 EUR an.

Damit kann dann wie folgt gerechnet werden. Als Basis sind hier nur Zahlungen
aus EU-L{\"a}ndern sowie ein Umsatz unter 5.000 EUR herangezogen worden:

\begin{center}
\begin{tabular}{llllll}
Kundenbetrag & Anteil & Geb{\"u}hr & Summe & Verbleibend\\
\hline
				&		 &		   &       &  \\
0,99			& 0,02	 & 0,35	   & 0,37  & 0,62\\
1,99			& 0,04	 & 0,35	   & 0,39  & 1,60\\
2,99			& 0,06	 & 0,35	   & 0,41  & 2,58\\
3,99			& 0,08	 & 0,35	   & 0,43  & 3,56\\
5,99			& 0,12	 & 0,35	   & 0,47  & 5,52\\
9,99			& 0,19	 & 0,35	   & 0,54  & 9,45\\
\hline
\end{tabular}
\end{center}
Was mir nicht m{\"o}glich war in Erfahrung zu bringen wie hoch die Stornogeb{\"u}hren
aussehen.\\
Es sei aber darauf hingewiesen das es gerade gegen PayPal recht laute
Gegenstimmen gibt, da es f{\"u}r Verk{\"a}ufer unter Umst{\"a}nden sehr unangenehm
werden kann wenn nicht alles glatt l{\"a}uft. Die Bandbreite reicht von
eingefrorenen Geldern ueber nur anteilig Ausgezahlte Einnahmen. Hier sei
dringend angeraten die Bedingungen von PayPal vor einem Abschluss zu lesen und
daf{\"u}r zu sorgen das alles korrekt abl{\"a}uft. Nach den Berichten welche
ich gepr{\"u}ft habe, sind die Konten tempor{\"a}r aufgrund des Geldw{\"a}schegesetzes
gesperrt worden. Einzelpersonen haben Zahlungen {\"u}ber das f{\"u}r Einzelpersonen
festgelegte Limit erhalten und damit die Pr{\"u}froutinen von PayPal aktiviert.
Wenn das Konto direkt bei Erstellung als Gesch{\"a}ftskonto deklariert wird,
sollten solche Schwierigkeiten nicht auftreten.

\section {ClickandBuy}
Neben PayPal ist ClickandBuy ein grosser Anbieter von Bezahlsystemen und wird
oft als der zweitgr{\"o}sste in diesem Gesch{\"a}ft genannt. Die Bezahlung als
solches ist f{\"u}r den Kunden schnell und einfach m{\"o}glich, auch hier ist
ein separates Kundenkonto notwendig.

Grunds{\"a}tzlich verlangt ClickandBuy eine einmalige Anmeldegeb{\"u}hr in
H{a}he von 19,95 EUR. Dazu kommen dann noch jeden Monat eine Grundgeb{\"u}hr von 19,95 EUR.

Sollte es fuer ClickandBuy nicht m{\"o}glich sein das Geld von Kunden zu erhalten,
zum Beispiel wegen fehlender Kontodeckung, werden f{\"u}r den Betreiber
Stornogeb{\"u}hren f{\"a}llig. Aktuell betragen diese 6 EUR.

Folgende Zahlen beziehen sich jedoch nur auf normale Transaktionen.
ClickandBuy weist daraufhin das diese Zahlen nicht f{\"u}r folgende Bereiche gilt:
Erotik, Dating, Music Download, Skillgaming und MMORPG
Da Browserspiele und damit Insulae auch unter die letzte Kategorie fallen,
k{\"o}nnen die folgenden Zahlen stark abweichen, sie sind daher nur als Richtwert
genannt.

Pro Tranaktion werden 0,35 EUR f{\"a}llig plus 2,9\% der Transaktionshoehe.

Dies sieht tats{\"a}chlich dann wie folgt aus:

\begin{center}
\begin{tabular}{llllll}
Kundenbetrag & Anteil & Geb{\"u}hr & Summe & Verbleibend\\
\hline
				&		 &		   &       &  \\
0,99			& 0,03	 & 0,35	   & 0,38  & 0,61\\
1,99			& 0,07	 & 0,35	   & 0,41  & 1,58\\
2,99			& 0,09	 & 0,35	   & 0,44  & 2,55\\
3,99			& 0,12	 & 0,35	   & 0,47  & 3,52\\
5,99			& 0,18	 & 0,35	   & 0,53  & 5,46\\
9,99			& 0,29	 & 0,35	   & 0,64  & 9,35\\
\hline
\end{tabular}
\end{center}

Diese unter Verbleibend stehenden Werte sind auch noch nicht steuerlich
betrachtet. Nachzulesen in den Bedingungen:
\begin{center} 
\href{http://www.clickandbuy.com/DE/de/merchantportal/registration/condition_digital.html}{http://www.clickandbuy.com}
\end{center}

\section {twofish}
Genauere Betrachtung steht noch aus. Nachzulesen unter:
\begin{center} 
\href{http://www.twofish.com/}{http://www.twofish.com}
\end{center}

\section {PlaySpan}
Genauere Betrachtung steht noch aus. Nachzulesen unter:
\begin{center}
\href{http://corp.playspan.com/}{http://corp.playspan.com}
\end{center}

%********** End of chapter **********